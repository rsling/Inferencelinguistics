\chapter{Scientific Inference and Error}
\label{sec:scientificinferenceanderror}

\section{Not up to the Task}
\label{sec:notuptothetask}

We are neither philosophers of science nor statisticians.
Hence, we keep our mouths shut as much as possible about matters that others (say, philosophers of science and statisticians) have conducted wars.
Reading a book and a handful of papers does not make anyone a philosopher or a statistician, and we won't fall into the trap of embarrassing ourselves by brining up half-digested bits of philosophy and statistics.


\section{Uncertainty}
\label{sec:uncertainty}

The world is full of people who are certain of many matters.
Or even all matters.
And then some.
Since this is not a journalistic or popular science book, I'm not exactly contemplating beliefs about politics, religion, social phenomena, etc.
I have in mind ordinary linguists who, I'm sure, would call themselves corpus linguists, empirically-oriented cognitive linguists, experimental psycholinguists, data-aware theoretical linguists, or whatever sounds good.
Many of them are fiercly certain about almost every single linguistic matter they have an interest in.
During a regular 15-minute question period after a linguistics talk, I once took the liberty of counting how often a specific colleague used the phrase \textit{I believe that} or \textit{Well, I believe that} with that familiar jovial stress on \textit{I}.
While asking three questions, the colleague in question used the phrase twelve times.
Such beliefs might not be held with the same fervor as religious or political beliefs, but they obviously guide the scientific practice of us researchers.
Why else would it be so important to reiterate them?
A pronouncedly more serious case I encountered with yet another colleague during a defense talk given by one of his doctoral students.
The student had not taken into account a possible secondary reading of a rather quirky sentence taken from a corpus, a fact which I pointed out, adding that nobody could be certain that whoever wrote the sentence was even a native speaker of English.
I was especially sceptical because the sentence was taken from a corpus I had built myself, and I was more than aware of the noisy nature of the data.
The esteemed colleague jumped to the rescue of his student, claiming that \textit{he was certain that the writer of the sentence must have had the correct reading in mind}.
Everybody nodded, and science progressed.
While this might be discarded by some (not including myself) as protective behaviour in a stressful situation, the claim was rendered even more absurd as the colleague is someone known for data-driven methods, being a key figure in the corpus-linguistic side of the empirical turn in the early 2000s.

These examples are of course anecdotal, which is the nature of examples and vignettes.


\section{Detecting Errors}
\label{sec:detectingerrors}
