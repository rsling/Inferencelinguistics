\chapter{Scientific Inference and Error}
\label{sec:inference}

\section*{Overview}

This chapter will be kept very short.
The main reason is that we're not even remotely close to being philosophers of science or philosophers of statistics.
However, that's what we'd have to be in order to write a more complete, lucid, and erudite chapter on the foundations of scientific inference.
In fact, we do not intend to make fools of ourselves by pretending to know anything about the philosophy of science based on half-digested books and articles from the Stanford Encyclopedia of Philosophy.
This chapter is merely account of our motivation to dip our feet into the murky waters of statistics in the first place.
It's the story of how we came to think that scientific practitioners should pay attention to the philosophy of science, the philosophy of science, and statistical inference in the first place.
Recipe-like application of experimental paradigms, paired with mindless statistical analysis is junk science.
Take it from us:
Realising that you've done it yourself is an utterly purifying experience.
Here we go!

In Section~\ref{sec:popsamples}

Then, in Section~\ref{sec:error}

Section~\ref{sec:jungle}

Finally, Section~\ref{sec:overview}

\section{From Populations to Samples}\label{sec:popsamples}

% \Key{population}
%
% \Key{sample}
%
% \Key{inference}

\section{Error}\label{sec:error}

\section{The Inference Jungle}\label{sec:jungle}

% \Key{Texas Marksman}
%
% \Key{validity}
%
% \Key{study design}
%
% \Key{pre-registration}
%
% \Key{meta-analyses}
%
% \Key{replication}
%
% \Key{open science}

\section{Overview of the Book}\label{sec:overview}

In terms of our statistical philosophy, we pretend to be Fisherian frequentists from Chapter~\ref{sec:fisher} to Chapter~\ref{sec:anova}.
In Chapter~\ref{sec:powerseverity}, we morph into Neyman-Pearson frequentists before turning into modellers with a wholistic frequentist background starting with Chapter~\ref{sec:correlation}.

To summarise the book in more traditional terms, Chapter~\ref{sec:fisher} introduces \textit{Fisher's Exact Test}.
Then, in Chapter~\ref{sec:describing} we introduce some \textit{descriptive statistics} (mostly measures of central tendency and variance), before discussing \textit{Confidence Intervals}---which we call \textit{Error Intervals}---in Chapter~\ref{sec:confidence}.
Tests of mean differences are the subject of Chapter~\ref{sec:zandt} (\textit{z-Test} and \textit{t-Test}) and Chapter~\ref{sec:anova} (\textit{ANOVA}).
Chapter~\ref{sec:powerseverity} is dedicated to \textit{Power} and the statistical notion of \textit{Severity} as developed by Deborah Mayo and Aris Spanos.
In Chapter~\ref{sec:nonparametric} we finish our discussion of classical texts with some \textit{non-parametric tests}.
Chapter~\ref{sec:correlation} lays the groundwork for our discussion of models by introducing \textit{Correlations}.
With Chapter~\ref{sec:lms}, we begin our discussion of models with \textit{Linear Models}, moving on to \textit{Generalised Linear Models} in Chapter~\ref{sec:glms} and finishing with \textit{Generalised Linear Mixed Models} in Chapter~\ref{sec:glmms}.



